%%==================================================
%% chapter03.tex for BIT Master Thesis
%% modified by yang yating
%% version: 0.1
%% last update: Dec 25th, 2016

%% modified by Meng Chao
%% version: 0.2
%% last update: May 29th, 2017
%%==================================================
\chapter{单目SLAM算法研究}
\label{chap:ALGORITHM}

%3.1
%\section{引言}
同步定位与地图重建算法利用多视图几何原理\upcite{[]},根据获取的图像信息估计摄像头在陌生环境中的位姿,构建环境地图。如果获取图像信息时,仅使用一个摄像头,则称为单目SLAM。本章主要介绍经典视觉SLAM算法框架;研究基于该框架的两种主流单目SLAM算法,在相同数据集上对比分析两种算法的优缺点,结合无人机运动特性选择合适的SLAM算法。

%3.1
\section{经典视觉SLAM算法框架}
SLAM算法实际是状态估计问题,将传感器数据抽象为适于估计的数学模型,通过观测模型与运动模型估计系统状态。经典视觉SLAM算法框架一般包括3个部分:视觉里程计,后端优化和闭环检测。视觉里程计估计帧间运动和局部地图,把传感器数据抽象为适于估计的数学模型;后端优化则接收视觉里程计估计的不同时刻位姿和回环检测信息,根据观测模型和运动模型建立约束关系,估计优化相机位姿与地图点位置,得到全局一致的轨迹与地图;回环检测判断相机是否曾经到达过当前位置,如果检测到回环反馈则给后端进行优化处理。算法框架如图\ref{fig3.1}所示,经典的视觉SLAM算法框架是过去十几年研究者们总结的成果,框架本身和所包含的算法已经较为成熟。本文研究内容主要是在该框架的基础上,针对单目SLAM算法存在的问题进行改进和完善。
%图3.1 高翔论文图1.3
\begin{figure}
\label{fig3.1}

\end{figure}
%3.1.1
\subsection{视觉里程计}
SLAM是一个状态估计问题,然而实际的传感器输出大都很难直接作为状态估计模型的输入。因而视觉里程计的主要任务是根据传感器输入,估计图像$I_i \rightarrow I_j$的相对位姿$T_ij$,串联相对位姿并三角化观测到的地图点得到运动轨迹与局部地图,将图像数据抽象为适于估计的数学模型。视觉里程计一般只估计相邻时刻的运动,和再过往的状态没有关联。视觉里程计主要包括两个部分,传感器数据关联和运动估计,其算法流程如图\eqref{fig3.2}所示。根据视觉里程计传感器关联数据方式的不同,当前主流SLAM算法分为基于直接法和基于特征的SLAM算法,具体原理将在3.2节中详细介绍。
%图3.2 vo综述中的vo流程。
\begin{figure}
\label{fig3.2}

\end{figure}

视觉里程计得到的轨迹和地图,由于在运动估计时只考虑了帧间的信息,每次估计都带有一定的误差,串联的轨迹无可避免会出现累计漂移,这将导致无法得到全局一致的轨迹与地图。需要通过后端优化算法进行处理,估计运动和周围环境空间的不确定性,减小估计误差,提高轨迹和地图的一致性。


%3.1.2
\subsection{后端优化}
SLAM算法的后端优化将SLAM问题看作最大后验概率估计问题,且大都采用因子图的方式来推到变量之间的以来关系。一般的,假设代估计状态变量为$\mathcal{X}$,在SLAM中变量$\mathcal{X}$包括无人机的轨迹和地图点的位置。传感器可以获得一组测量值$Z=\lbrace z_k:k=1,\ldots ,m\rbrace$,且每个测量值可以表示为状态变量$\chi$的函数,例如$z_k=h_k\left( \mathcal{X}_k \right)+\epsilon_k$,其中$\mathcal{X}_k \in \mathcal{X}$是变量的子集,$h_k(\cdot)$表示传感器的观测模型,$\epsilon_k$是随机观测误差。

在最大后验概率估计中,通过计算变量$\mathcal{X}^*$的概率分布来估计状态变量$\mathcal{X}$,$\mathcal{X}^*$表示最大后验概率$\mathds{P}\left(\mathcal{X} \vert Z\right)$:
\begin{equation}
\label{equ3.1}
\mathcal{X}^* 
\doteq 
\argmax \limits_{\mathcal{X}} \mathds{P}\left(\mathcal{X} \vert Z\right) 
=
\argmax \limits_{\mathcal{X}}\mathds{P}\left(Z \vert \mathcal{X}  \right)\mathds{P}\left(\mathcal{X}\right)
\end{equation}
上式推导来源于贝叶斯定义。在公式\eqref{equ3.1}中,$\mathds{P}\left(Z \vert \mathcal{X}  \right)$表示在状态变量$\mathcal{X}$分布确定情况下测量值$Z$的似然,$\mathds{P}\left(\mathcal{X}\right)$表示状态$\mathcal{X}$的先验概率。先验概率包括任何关于状态$\mathcal{X}$的先验信息,如果没有可用的先验信息,则$\mathds{P}\left(\mathcal{X}\right)$表示为一个常亮从而可以从优化过程中移除。在这种情况下,最大后验概率可以简化为极大似然估计。

假设测量值$Z=\lbrace z_k:k=1,\ldots ,m\rbrace$相互独立(影响测量的噪声是不相关的),公式\eqref{equ3.1}可以因式分解为:
\begin{equation}
\label{equ3.2}
\mathcal{X}^* 
\doteq 
\argmax \limits_{\mathcal{X}} \mathds{P}\left(\mathcal{X}\right) \prod \limits_{k=1}^{m} \mathds{P}\left(z_k \vert \mathcal{X}  \right)
=
\mathds{P}\left(\mathcal{X}\right) \prod \limits_{k=1}^{m} \mathds{P}\left(z_k \vert \mathcal{X}_k  \right)
\end{equation}
其中,等式右边的测量值$z_k$仅仅与状态变量$\mathcal{X}_k$的子集有关。

公式\eqref{equ3.2}可以因子图表示。待估计的状态变量用节点表示,似然$\mathds{P}\left(z_k \vert \mathcal{X}_k  \right)$和先验$\mathds{P}\left(\mathcal{X} \right)$由因子表示,因子建立了对节点子集的概率约束。因子图是一种建立第$k$个因子和对应状态变量$\mathcal{X}_k$依赖关系的图模型。因子图可以将SLAM问题可视化,便于理解,一个简单SLAM问题的因子图描述如图\ref{fig3.3}表示。
% 综述因子图
\begin{figure}
\label{fig3.3}

\end{figure}



为了更清楚的表示公式\eqref{equ3.2},假设观测模型的测量噪声$\epsilon_k$服从信息矩阵为$\Omega_k$的0均值高斯分布。测量值的似然可以表示为
\begin{equation}
\label{equ3.3}
\mathds{P}\left(z_k \vert \mathcal{X}_k  \right) \varpropto \exp ( -{1 \over 2} \left\| h_k\left( \mathcal{X}_k \right) - z_k \right\|_{\Omega_k}^2 )
\end{equation}
其中,使用符号$\left\| e \right\|_{\Omega}^2 = e^T \Omega e$。类似的,假设先验也服从高斯分布:
\begin{equation}
\label{equ3.4}
\mathds{P}\left( \mathcal{X}_k  \right) \varpropto \exp ( -{1 \over 2} \left\| h_0\left( \mathcal{X}_0 \right) - z_0 \right\|_{\Omega_0}^2 )
\end{equation}
其中$h_0(\cdot)$表示观测模型,先验均值为$z_0$,信息矩阵为$\Omega_0$。由于最大后验与最小负对数后验等价,则公式\eqref{equ3.2}的最大后验估计可以表示为:
\begin{equation}
\label{equ3.5}
\mathcal{X}^* 
=
\argmin \limits_{\mathcal{X}} - \log \left( \mathds{P}\left(\mathcal{X}\right) \prod \limits_{k=1}^{m} \mathds{P}\left(z_k \vert \mathcal{X}_k  \right) \right)
=
\argmin \limits_{\mathcal{X}} \sum \limits_{k=0}^{m} -{1 \over 2} \left\| h_k\left( \mathcal{X}_k \right) - z_k \right\|_{\Omega_k}^2
\end{equation}
公式\eqref{equ3.5}的形式是典型的非线性最小二乘问题,其中$h_k{\cdot}$是一个非线性函数。需要注意的是,公式\eqref{equ3.5}推导的前提是假设观测模型的观测噪声服从高斯分布。对于不同的观测噪声分布,将会得到不同的优化目标函数。例如,若观测噪声服从拉普拉斯分布,则公式\eqref{equ3.5}的平方$l_2$范数应变为$l_1$范数。为了增加算法对于离群点的鲁棒性,通常使用鲁棒核函数替换公式\eqref{equ3.5}中的平方$l_2$范数。

公式\eqref{equ3.5}的形式与运动结构学中的Bundle Adjustment(BA)问题相似。这是因为公式\eqref{equ3.5}与BA都是从最大后验概率估计出发推导得到的。但是,SLAM问题具有两个特点。首先,不同于BA问题受限于几何模型约束,公式\eqref{equ3.5}适用于各种传感器模型。比如惯性传感器,GPS,轮式传感器等。其次,公式\eqref{equ3.5}是增量式的求解,在无人机运动时,可以获得新的观测值。

对于解决公式\eqref{equ3.5}的最小化问题,可通过连续线性化进行求解。例如高斯-牛顿法(G-N)和列文伯格-马夸尔特法(L-M)。高斯-牛顿方法从初始估计给定的初值$\hat{\mathcal{X}}$开始迭代,每次迭代时高斯-牛顿法估计公式\eqref{equ3.5}的最小值
\begin{equation}
\label{equ3.6}
\delta_\mathcal{X}^* 
=
\argmin \limits_{\delta_\mathcal{X}} {1 \over 2} \sum \limits_{k=0}^{m} \left\| A_k\delta_\mathcal{X}-b_k \right\|_{\Omega_k}^2
=
\argmin \limits_{\delta_\mathcal{X}} {1 \over 2} \left\| A_k\delta_\mathcal{X}-b \right\|_{\Omega}^2
\end{equation}
其中公式\eqref{equ3.6}中的$\delta_\mathcal{X}$表示初始估计状态$\hat{\mathcal{X}}$的微小修正量。$A_k \doteq {\partial h_k(\mathcal{X}) \over \partial \mathcal{X}} $是观测模型$h_k(\cdot)$关于$\mathcal{X}$的雅各比,$b_k \doteq z_k-h(\mathcal{X})$表示测量值与观测模型输出的残差;公式\eqref{equ3.6}右边的矩阵$A$,$b$是由$A_k$,$b_k$组成的矩阵;$\Omega$是由观测噪声信息矩阵$\Omega_k$组成的对角块矩阵。

最小化公式\eqref{equ3.6}对应的微小修正$\delta_\mathcal{X}^* $可以用下面的闭型公式计算
\begin{equation}
\label{equ3.7}
\delta_\mathcal{X}^* = - \left( A^T \Omega A \right)^{-1} A^T \Omega b
\end{equation}
每次通过$\hat{\mathcal{X}} \leftarrow \hat{\mathcal{X}}+\delta_\mathcal{X}^*$迭代更新估计状态,矩阵$A^T \Omega A$称为Hessian矩阵。之前的推导过程中,我们假设$\mathcal{X}$属于向量空间。如果$\mathcal{X}$属于光滑的流形空间(如旋转),则高斯-牛顿法的形式保持不边,但是迭代更新方程$\hat{\mathcal{X}} \leftarrow \hat{\mathcal{X}}+\delta_\mathcal{X}$会被更合适的映射法则取代。在机器人领域中,通常使用符号$\oplus$表示状态更新的映射关系,并且将状态的微小修正量$\delta_\mathcal{X}$定义在流形状态变量$\hat{\mathcal{X}}$的切空间上,此时的更新方程为$\hat{\mathcal{X}}  \leftarrow \hat{\mathcal{X}} \oplus \delta_\mathcal{X}$。

现代SLAM算法最大的进步,在于认识到公式\eqref{equ3.7}中的雅各比矩阵$A$的稀疏性,而矩阵$A$的稀疏性是由于因子图内在的拓扑逻辑所决定的,这可以利用线性方法快速求解估计状态的微小增量$\delta_\mathcal{X}^*$。此外,利用矩阵$A$的稀疏性可以设计增量式的求解方法,将更新后的状态变量作为新的观测值。当前主流SLAM后端优化库(如GTSAM,g2o,Ceres,iSAM,SLAM++)可以在几秒之内求解数以万计的状态变量。

目前,通常使用最大后验估计,因子图优化,图优化,完全平滑和平滑映射来描述SLAM问题。其中特别需要介绍的描述方法是位姿图优化,位姿图中的状态变量是无人机运动中的采样位姿,约束是两位姿之间的相对运动约束。

%3.1.3
\subsection{回环检测}

回环检测又称闭环检测,主要解决SLAM问题中位姿估计的轨迹随时间漂移的问题。





%3.2
\section{主流单目视觉SLAM算法研究}

%3.2.1
\subsection{基于直接法的SLAM算法}

\subsection*{数据关联}

\subsection*{运动估计}



The current frames will replace the current key frames according with the Direct SE(3) image alignment. An existing key frame ${k_i} = ({I_i},{D_i},{V_i})$, the relative posture ${\xi _{ji}} \in se(3)$ of a new image ${I_j}$ is obtained by minimizing the variance-normalized photometric error

\begin{equation}
E_{p}(\xi _{ji}) = \sum\limits_{p \in \Omega _{D_{i}}} \left\|  \frac{r_{p}^{2} (p,\xi_{ji})}{\sigma_{r_{p}}^{2} (p,\xi_{ji})}  \right\|_{\delta}
\end{equation}


\begin{equation}
r_{p}(p,\xi _{ji}): = I_{i}(p) - I_{j}(\omega (p,D_{i}(p),\xi _{ji}))
\end{equation}

\begin{equation}
\sigma _{r_{p}} ^{2}(p,\xi _{ji}): = 2\sigma _{I} ^{2} + {\left( \frac {\partial r_{p}(p,\xi_{ji})} {\partial D_{i} (p)} \right)^{2}} V_{i}(p)
\end{equation}

The formula (13) is the Huber norm and applied to the normalized residual.

\begin{equation}
{\left\| r^{2} \right\|_{\delta}}:=\left\{\begin{array}{ll}
\frac {r^{2}} {2\delta}            & |r|\leq \delta       \\
|r|-\frac {\delta} {2}      & |r| > \delta
\end{array} \right.
\end{equation}




%3.2.2
\subsection{基于特征的SLAM算法}

\subsection*{数据关联}

\subsection*{运动估计}


If the camera moves too far away from the existing map, a new key frame is created from the most recent tracked image. We threshold a weighted combination of relative distance and angle to the current key frame

\begin{equation}
dist({\xi _{ji}}): = \xi _{ji}^TW{\xi _{ji}}
\end{equation}

Where W is a diagonal matrix containing the weights. Note that, as described in the following section, each key frame is scaled such that its mean inverse depth is one. This threshold is therefore relative to the current scale of the scene, and ensures sufficient possibilities for small-baseline stereo comparisons.

Once a new frame is chosen to become a key frame, its depth map is initialized by projecting points from the previous key frame into it, followed by one iteration of spatial regularization and outlier removal [5]. Afterwards, the depth map is scaled to have a mean inverse depth of one -this scaling factor is directly incorporated into the sim(3) camera pose. Finally, it replaces the previous key frame and is used for tracking subsequent new frames.



%3.2.3
\subsection{两种SLAM算法比较}


In contrast to RGB-D or Stereo SLAM, monocular SLAM is inherently scale-ambivalent, the absolute scale of the world is not observable. Over long trajectories this leads to scale-drift, which is one of the major sources of error [9]. Further, all distances are only defined up to scale, which causes threshold-based outlier rejection or parametrized robust kernels to be ill-defined.

We select an advanced method in [1] for better perform direct, scale-drift image alignment, which is used to align two differently scaled key frames. For the photometric residual ${r_p}$, the depth residual ${r_d}$ which penalizes deviations in inverse depth between key frames, allowing to directly estimate the scaled transformation between them. The total error function becomes

\begin{equation}
E({\xi _{ji}}): = {\sum\limits_{p \in {\Omega _{{D_i}}}} {\left\| {\frac{{r_p^2(p,{\xi _{ji}})}}{{\sigma _{{r_p}(p,{\xi _{ji}})}^2}} + \frac{{r_d^2(p,{\xi _{ji}})}}{{\sigma _{{r_d}(p,{\xi _{ji}})}^2}}} \right\|} _\delta }
\end{equation}
the depth residual is computed

\begin{equation}
r_{d}(p,\xi _{ji}): = [p^{'}]_{3} - D_{j}([p^{'}]_{1,2})
\end{equation}

% 公式换行用 \\ 
\begin{equation}
\begin{split}
\sigma_{r_{d}(p,\xi_{ji})} ^{2} :=V_{j}( [ p^{'} ]_{1,2} )  {\left( \frac {\partial r_{d}(p,\xi_{ji})} {\partial D_{j} \left[ p^{'} \right]_{1,2}}  \right)^{2}} + V_{i}(p) \left( \frac {\partial r_{d}(p,\xi_{ji})} {\partial D_{j}(p)} \right)
\end{split}
\end{equation}

After a new key frame ${k_i}$ is added to the map, a number of possible loop closure key frames ${k_{ji}},....{k_{jn}}$ is collected. We use the closet ten key frames, as well as a suitable candidate. To avoid insertion of false or falsely tracked loop closures, we then perform a reciprocal tracking check. For each candidate ${k_{jk}}$ we independently track ${\xi _{{j_k}i}}$ and ${\xi _{i{j_k}}}$.

\begin{equation}
e\!(\!{\xi _{{j_k}i}},{\xi _{i{j_k}}})\!: \!= \!{\!(\!{\xi _{{j_k}i}} \!\circ \!{\xi _{i{j_k}}})\!^T}{\!(\!{\Sigma _{{j_k}i}}\! +\! Ad{j_{{j_k}i}}{\Sigma _{i{j_k}}}Adj_{{j_k}i}^T\!)\!^{ - 1}}\!(\!{\xi _{{j_k}i}}\! \circ\! {\xi _{i{j_k}}}\!)
\end{equation}

Only if the two estimates are statistically similar, if formula (18) is sufficiently small, they are added to the global map. For this, the adjoint $Ad{j_{{j_k}i}}$ is used to transform ${\Sigma _{i{j_k}}}$ into the correct tangent space.




%3.3
\section{本章小结}




The map is represented as a pose graph of keyframes: Each keyframe $\kappa_{i}$ consists of a camera image $I_{i}\!:\!\Omega_{D_{i}}\!\rightarrow \!{\mathbb{R}}$,an inverse depth map $D_{i}\!:\!\Omega_{D_{i}}\!\rightarrow \!{\mathbb{R}}^{+}$, and the variance of the inverse depth $ V_{i}\!:\!\Omega_{D_{i}}\!\rightarrow \!{\mathbb{R}}^{+}$.Note that the depth map and variance are only defined for a subset of pixels $ \Omega_{D_{i}}\!\subset\!\Omega_{i} $, containing all image regions in the vicinity of sufficiently large intensity gradient, hence semi-dense.Edges $ \varepsilon_{ji}$ between keyframes contain their relative alignment as similarity transform $ \xi_{ji}\in sim(3) $, as well as the corresponding covariance matrix $\Sigma_{ji} $

The map, consisting of a set of keyframes and tracked sim(3)-constraints, is continuously optimized in the background using pose graph optimization. The
error function that is minimized is in accordance with the left-multiplication convention
\begin{equation}
E(\xi\!_{W\!_{1}}\!...\!\xi\!_{W\!_{n}})\!:\!= \sum\limits_{(\!\xi\!_{ji},\Sigma\!_{ji}\!)\!\in\!\varepsilon}\!(\!\xi_{ji}\!\circ\!\xi_{\!W_{i}\!}^{-1}\! \!\circ\!\xi_{\!W\!_{j}}\!)\!^{T}\!\Sigma_{\!j\!i}^{\!-1\!}\!(\!\xi_{ji}\!\circ\!\xi_{\!W_{i}\!}^{-1}\! \!\circ\!\xi_{\!W\!_{j}}\!)
\end{equation}


